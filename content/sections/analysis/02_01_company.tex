\subsection{Общая характеристика и основные направления деятельности ООО \enquote{Сампад}}
\label{sec:develop:company}

Компания Sampad основана в 2013 году. Полное название предприятия: общество с ограниченной ответственностью \enquote{Сампад}. Компания является разработчиком заказного программного обеспечения и поставщиком ИТ-услуг  для компаний из различных стран. Штаб-квартира компании расположена в Минске, также есть офис в США. Штат сотрудников насчитывает более 80 инженеров и IT-консультантов \cite{sampad_data}.
Sampad имеет большой опыт в таких областях, как:
\begin{itemize}
    \item нативные мобильные приложения (iOS: Obj C/Swift, Android: Java/Kotlin);
    \item кроссплатформенные мобильные приложения (React Native, Cordova Ionic, Flutter);
    \item финансовые бизнес-приложения, Fintech (Python, ASP .Net);
    \item IoT приложения;
    \item поддержка бизнес-приложений.
\end{itemize}
Преимуществами сотрудничества с Sampad пользуются десятки компаний из различных секторов экономики, в том числе:

\begin{itemize}
    \item банки и финансовые компании;
    \item розничная торговля и потребительские товары;
    \item информационный и медиа-бизнес;
    \item индустрия путешествий;
    \item образование;
    \item медицина;
    \item сельское хозяйство;
    \item страхование;
    \item спорт;
    \item рекрутинг;
    \item автобизнес.
\end{itemize}
Основными заказчиками являются: Microsoft, Oracle, The Coca-Cola Company, McDonalds, Nestle, Timotei и многие другие.

Компания делает большой акцент на проектах с использованием искусственного интеллекта (машинное обучение, нейронные сети и др.). Существуют проекты в сфере образования, сбора финансовой информации, кредитования малого и среднего бизнеса. А также приложения для сельского хозяйства (предсказание появления насекомых на полях).

Индустрия электронного светового оборудования и рынок IoT занимают особое место в деятельности кампании Sampad. Первый проект в данной сфере появился еще в 2016 году. На данный момент существует уже 4 проекта со схожей тематикой и еще несколько находятся на стадии проектирования. Компания разрабатывает как мобильные и веб приложения для управления световым оборудованием, так и прошивку для самого оборудования (C/C++, Arduino). Также компания консультирует заказчиков по поводу электронных компонентов, находящихся в данном оборудовании, хоть и не отвечает непосредственно за разработку электронных схем.