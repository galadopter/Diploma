\newpage
\subsection{Информационная модель системы}
\label{sec:develop:umlDiagrams}

При проектировании системы для поиска, вызова и оплаты такси были выделены такие сущности, как:
\begin{itemize}
	\item Анимация;
	\item Лампочка;
	\item Отображение;
	\item Сеть;
	\item Устройство (гирлянда);
	\item Пользователь.
\end{itemize}

Представление информационной модели в графическом виде приведено на диаграмме Б.1 приложения Б. 

Сущность Анимация необходима для отображения стандартных и пользовательских анимаций в приложении и содержит в себе следующие атрибуты:
\begin{itemize}
	\item name~-- имя анимации;
	\item id~-- id анимации, используется при отправке анимации на гирлянду (0~-- посылается через байты, 1, 2,...~-- запускает анимацию внутри самой гирлянды);
	\item previewType~-- тип анимации (default, custom);
	\item totalTime~-- полное время анимации;
	\item speed~-- скорость анимации;
	\item type~-- тип отображения в приложении (string, tree, custom, grid);
	\item createdDate~-- дата создания;
	\item currentColorIndex~-- выбранный набор цветов;
	\item maximumColorsCount~-- максимальное количество цветов в наборе;
	\item maximumSpeed~-- максимальное значение скорости;
	\item minimumSpeed~-- минимальное значение скорости;
	\item intensity~-- интенсивность анимации;
	\item lamps~-- лампочки анимации;
	\item editOptions~-- доступные опции для редактирования;
	\item colors~-- наборы цветов.
\end{itemize}

Сущность Лампочка необходима для представления лампочки на гирлянде под определенным адресом в конкретный момент времени и содержит в себе следующие атрибуты:
\begin{itemize}
	\item id~-- id лампочки;
	\item address~-- адрес лампочки;
	\item color~-- цвет лампы;
	\item coordX~-- координата по оси абсцисс;
	\item coordY~-- координата по оси ординат;
	\item deltaT~-- момент времени, когда лампочка должна загореться.
\end{itemize}

Сущность Отображение необходима для представления гирлянды на экране приложения и содержит в себе следующие атрибуты:
\begin{itemize}
	\item name~-- имя отображения;
	\item id~-- id отображения;
	\item totalTime;
	\item type~-- тип отображения;
	\item lamps~-- лампочки отображения.
\end{itemize}

Сущность Сеть необходима для представления сети из гирлянд и содержит в себе следующие атрибуты:
\begin{itemize}
	\item name~-- имя сети;
	\item animationName~-- имя анимации, проигрываемой на сети;
	\item animationState~-- тип отображения анимаций в сети;
	\item devices~-- устройства в сети.
\end{itemize}

Сущность Устройство необходима для представления гирлянды в сети и содержит в себе следующие атрибуты:
\begin{itemize}
	\item id~-- id устройства;
	\item name~-- имя устройства;
	\item ip~-- ip-адрес устройства;
	\item port~-- порт в сети;
	\item mac~-- mac-адрес устройства;
	\item animationName~-- проигрываимая анимация на устройстве;
	\item lampsType~-- тип лампочек на устройстве;
	\item isChosen~-- флаг выбрано ли устройство для анимирования в сети.
\end{itemize}

Сущность Пользователь необходима для представления пользователя для отправки анимаций и содержит в себе следующие атрибуты:
\begin{itemize}
	\item phoneNumber~-- номер телефона пользователя;
	\item email~-- email-адрес пользователя.
\end{itemize}