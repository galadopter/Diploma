\subsection{Описание алгоритмов работы подсистемы}
\label{sec:develop:algorithms}

\subsubsection{} Алгоритм аппроксимации
\label{sec:develop:algorithms:approximation}

При выполнении калибровки адресных светодиодных лент, после обработки изображения и группировки лампочек, происходит важный этап~--- аппроксимация. Так как алгоритм не находит 100\% лампочек, но пользователю нужно анимацию на всех, то применяется данный алгоритм. Он находит участки не найденных лампочек и выстраивает прямую между двумя крайними найденными лампочками. И на данную прямую равномерно наносятся не найденные лампочки. Алгоритм представлен в приложении Г на рисунке Г.1.

\subsubsection{} Алгоритм конвертации байт анимации
\label{sec:develop:algorithms:bytesSending}

При отправке анимации на адресную светодиодную ленту происходит ее конвертация в байты для оптимизации как самого процесса отправки (уменьшая ее время), так и процесс отображения анимации на адресной ленте, правильно компонуя байты лампочек друг с другом. В самом алгоритме лампочки сортируются по кадрам анимации (по deltaT) и разбиваются на порции по 50 лампочек, после этого они конвертируются в байты, группируются и отправляются на адресную ленту. Данный алгоритм представлен в приложении Г на рисунке Г.2.