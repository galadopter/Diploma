\subsection{Описание алгоритма калибровки адресной светодиодной ленты}
\label{sec:develop:algorithm}

При использовании адресных светодиодных лент пользователь может повесить ее каким угодно образом. Однако при этом часть анимаций, которые расчитаны на конкретное расположение лампочек относительно друг друга, будут воспроизводится некорректно, представляя из себя мешанину из цветов. Алгоритм калибровки призван устранить данный недостаток анимаций. Пользователь с помощью камеры снимает небольшую анимацию на адресной светодиодной ленте, а потом алгоритм обрабатывает данные снимки и понимает расположение лампочек относительно друг друга.

% Анимация калибровки
% \label{sec:develop:algorithm:animation}
Анимация калибровки представляет собой последовательность из трех цветов (красный, синий и зеленый) для каждой лампочки. Ее суть заключается в том, что каждой лампочке задают определенную последовательность из трех цветов (адреса представляются в троичной системе исчисления, 0~-- красный, 1~-- синий и 3~-- зеленый). Последовательности при каждой калибровке генерируются заново, чтобы исключить проблему, когда рядом лежащие лампы горят одним цветом, что может привести к наслоению их друг на друга при обработке изображения. Пример изображен на рисунке~\ref{fig:develop:algorithm:animation}.

Шаги анимации:
\begin{enumerate}[label=\arabic*]
	\item Все лампочки горят синим цветом~-- нужно для того, чтобы алгоритм понял, что анимация началась.
	\item Все лампочки гаснут~-- алгоритм использует прошлый кадр с синими лампами и этот кадр, чтобы вычленить из всего изображения место, где будут находится лампы, отфильтровав все остальные предметы с синим, красным или зеленым цветом.
	\item Несколько сгенерированных кадров~-- алгоритм генерирует несколько кадров анимации, используя три цвета (красный, синий и зеленый). Количество кадров в анимации вычисляется как $\lfloor\log_3 a\rfloor$, где $a$~-- количество лампочек в адресной ленте.
\end{enumerate}

Алгоритм калибровки:
\begin{enumerate}[label=\arabic*]
	\item Сгенерировать анимацию калибровки.
	\item Отправить анимацию на ленту.
	\item Поиск синего цвета.
	\item Сохранить изображение с камеры в буфер.
	\item Искать синие пиксели на изображении.
	\item Поиск синего цвета Пока синие пиксели меньше 5000.
	\item Запустить таймер по снятию изображений.
	\item Найти все группы синих пикселей.
	\item Сохранить изображение с камеры в буфер.
	\item Найти все группы синих пикселей.
	\item Найти разницу между первым и вторым изображениями (разница~-- лампочки в адресной ленте).
	\item Сохранение изображения с камеры в буфер.
	\item Найти на изображении группы пикселей из 3 цветов (красный, синий и зеленый).
	\item Отфильтровать группы на яркость и принадлежность к лампочками в адресной ленте.
	\item Сохранение изображения с камеры в буфер Пока количество сохранений не равно 6.
	\item Сгруппировать найденные цвета в последовательности.
	\item Соотнести найденные последовательности с сгенерированными.
	\item Вычислить индексы лампочек у каждой последовательности.
	\item Если количество найденных лампочек больше 30, то пункт \ref{enum:algorithm:approximation} иначе пункт \ref{enum:algorithm:calibrationError}.
	\item \label{enum:algorithm:approximation} Аппроксимировать полученный результат (достроить недостающие лампочки).
	\item \label{enum:algorithm:calibrationError} Показать сообщение об ошибке калибровки.
\end{enumerate}

~
\begin{figure}[H]
\centering
	\includegraphics[scale=0.8]{figures/calibration/animation.jpg}
	\caption{Пример кадра анимации}
	\label{fig:develop:algorithm:animation}
\end{figure}

% Ядро алгоритма
% \label{sec:develop:algorithm:core}
База алгоритма основывается на распознании цвета пикселей изображений и на Connected-Component Labeling алгоритме. Его суть состоит в группировке одинаковых пикселей с помощью алгоритма поиска в глубину. Так как этот алгоритм предназначен для работы с изображениями только двух цветов, то нам надо превратить изображение, снятое с камеры телефона, в черно-белое. Использовать черно-белое изображение для распознания трех цветов одновременно не представляется возможным, поэтому для начала мы формируем три изображения для каждого из цветов отдельно, переводя изображение в черно-белое, где белым выделен нужный нам цвет (Рисунок~\ref{fig:develop:algorithm:colorFinding}).

~
\begin{figure}[H]
\centering
	\includegraphics[scale=0.8]{figures/calibration/findedColor.jpg}
	\caption{Пример найденного цвета в кадре}
	\label{fig:develop:algorithm:colorFinding}
\end{figure}

Пример действия алгоритма Connected-Component Labeling представлен на рисунке \ref{fig:develop:algorithm:ccl_example}. Данный алгоритм группирует ближайшие пиксели белого цвета. На верхней картинке показаны не размеченные пиксели. На нижней картинке показан результат работы алгоритма~-- размеченные группы белых пикселей.

~
\begin{figure}[H]
\centering
	\includegraphics[scale=0.3]{figures/calibration/ccl_example.png}
	\caption{Пример работы CCL алгоритма}
	\label{fig:develop:algorithm:ccl_example}
\end{figure}

Изображение может быть некачественным, поскольку оно сжимается для лучшей скорости работы; также еще стоит учитывать покачивания при съемке, засвет от других источников и так далее. Следовательно, лампочка скорее всего будет состоять не из одной группы лежащих рядом пикселей, а из нескольких (Рисунок~\ref{fig:develop:algorithm:toBlackWhite}). Для этого применяется размытие по Гауссу вместе с фильтром по яркости (Рисунок~\ref{fig:develop:algorithm:blurring}). Далее находим группы из белых пикселей с помощью описанного выше CCL алгоритма (каждая группа - лампочка определенного цвета). При обработке всех кадров получаем последовательность из цветов с похожими координатами, так как это скорее всего одна и та же лампа, то группируем их в последовательность. Далее сопоставляем последовательность с уже заранее сгенерированными и находим адрес лампочки (Рисунок~\ref{fig:develop:algorithm:grouping}).

~
\begin{figure}[H]
\centering
	\includegraphics[scale=0.5]{figures/calibration/toBlackWhite.jpg}
	\caption{Первоначальное распознавание цвета}
	\label{fig:develop:algorithm:toBlackWhite}
\end{figure}

\begin{figure}[H]
\centering
	\includegraphics[scale=0.5]{figures/calibration/blurring.jpg}
	\caption{Распознание цвета после размытия}
	\label{fig:develop:algorithm:blurring}
\end{figure}

\begin{figure}[H]
\centering
	\includegraphics[scale=0.35]{figures/calibration/grouping.pdf}
	\caption{Пример группировки цветов по координатам}
	\label{fig:develop:algorithm:grouping}
\end{figure}

Сопоставление последовательности происходит следующим образом: сперва ищется точка с полностью совпадающей последовательностью цветов, если такой нет, то ищется последовательность отличающаяся на один цвет, потом на два и на три. Далее из всех найденных возможных точек, выбирается ближайшая к остальным уже найденным точкам.

% Аппроксимация
% \label{sec:develop:algorithm:approximation}

~
\begin{figure}[H]
\centering
	\includegraphics[scale=0.45]{figures/calibration/approximation.pdf}
	\caption{Пример работы аппроксимации (красной границей отмечены достроенные лампочки)}
	\label{fig:develop:algorithm:approximation}
\end{figure}

После обработки изображения часто остаются нераспознанными некоторые лампочки (обычно 20-30\%). Для устранения данной проблемы используется алгоритм аппроксимации. Он берет соседние найденные лампы возле ненайденных (например 63 и 69 лампочка, если алгоритм не нашел с 64 по 68 лампочку), строит между ними прямую и располагает на равноудаленных участках этого отрезка ненайденные лампочки. Если ненайденные лампочки находятся на концах адресной светодиодной ленты, то он просто строит примерно похожую линию и располагает лампочки на ней. Это помогает построить примерно похожую адресную ленту на экране телефона.
