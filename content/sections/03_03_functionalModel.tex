\subsection{Функциональная модель}
\label{sec:develop:functionalModel}

Функционирование украшения дома в данном проекте описаны с помощью функциональной модели IDEF0.

Для представления процессов, возникающих при приобретении и использовании гирлянд (украшений) была разработана функциональная модель "Украшение дома на праздники". Процесс состоит из нескольких этапов, которые подробно описаны на рисунках~\ref{fig:develop:functionalModel:main}~–~\ref{fig:develop:functionalModel:a42_connecting}.

~
\begin{figure}[H]
\centering
	\includegraphics[scale=0.6]{figures/functionalModel/main.jpg}
	\caption{Контекстная диаграмма процесса украшения дома}
	\label{fig:develop:functionalModel:main}
\end{figure}

На рисунке~\ref{fig:develop:functionalModel:main} представлена контекстная диаграмма верхнего уровня, входными данными для которой являются дом без украшений и пожелания к ним. В рамках процесса на выходе получается украшенный дом. Далее в соответствии со вторым и третьим принципами, распишем основное действие на уровни.

Украшение дома декомпозировано на 4 блока: поиск, покупка, установка и настройка украшений.

Функциональная модель украшения дома показана на рисунке~\ref{fig:develop:functionalModel:a0_decoration}.

~
\begin{figure}[H]
\centering
	\includegraphics[scale=0.6]{figures/functionalModel/a0_decoration.jpg}
	\caption{Диаграмма использования системы}
	\label{fig:develop:functionalModel:a0_decoration}
\end{figure}

Рассмотрим последовательно каждый блок.

Первый блок (рисунок~\ref{fig:develop:functionalModel:a1_search}):

Данный блок необходим для поиска подходящего украшения для дома. Данный процесс включает в себя вход в онлайн магазин, установку фильтров для поиска, просмотр результатов поиска, консультация с работником онлайн-магазина (для дополнительной фильтрации списка выбранных украшений) и, наконец, выбор подходящего варианта.

 ~
\begin{figure}[H]
\centering
	\includegraphics[scale=0.6]{figures/functionalModel/a1_search.jpg}
	\caption{Диаграмма использования системы}
	\label{fig:develop:functionalModel:a1_search}
\end{figure}

Второй блок представлен на рисунке~\ref{fig:develop:functionalModel:a2_buy}. Данный блок включает в себя добавление заказа (украшения) в корзину, ввод платежной информации, ввод информации о доставке (адрес, индекс и т.д.) и оплата товара.

 ~
\begin{figure}[H]
\centering
	\includegraphics[scale=0.6]{figures/functionalModel/a2_buy.jpg}
	\caption{Диаграмма использования системы}
	\label{fig:develop:functionalModel:a2_buy}
\end{figure}

 ~
\begin{figure}[H]
\centering
	\includegraphics[scale=0.6]{figures/functionalModel/a3_install.jpg}
	\caption{Диаграмма использования системы}
	\label{fig:develop:functionalModel:a3_install}
\end{figure}

Третий блок (рисунок~\ref{fig:develop:functionalModel:a3_install}):

Данный блок описывает процесс установки купленного украшения в доме. Данный процесс включает в себя распаковку украшения, чтение инструкции, выбор места для украшения и его установка.

 ~
\begin{figure}[H]
\centering
	\includegraphics[scale=0.6]{figures/functionalModel/a4_settings.jpg}
	\caption{Диаграмма использования системы}
	\label{fig:develop:functionalModel:a4_settings}
\end{figure}

Четвертый блок (рисунок~\ref{fig:develop:functionalModel:a4_settings}):

Данный блок описывает процесс настройки украшений. Данный процесс включает в себя загрузку приложения для управления украшениями, подключение их к приложению, калибровку и отправку уже откалиброванных анимаций на украшения. 

~
\begin{figure}[H]
\centering
	\includegraphics[scale=0.6]{figures/functionalModel/a42_connecting.jpg}
	\caption{Диаграмма использования системы}
	\label{fig:develop:functionalModel:a42_connecting}
\end{figure}

Декомпозиция четвертого блока (рисунок~\ref{fig:develop:functionalModel:a42_connecting}):

Данная декомпозиция подробнее рассматривает процесс подключения мобильного приложения к украшению (украшениям). Она разбивает данный процесс на следующие компоненты: подключение напрямую к украшению через Wi-Fi, добавление украшения к местному роутеру (ввод пароля к точке). Затем идет процес подключения украшения к роутеру. После всего следует поиск украшения в списке доступных устройств.