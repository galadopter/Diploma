\section{Описание индустрии светового оборудования и рынка мобильных приложений}
\label{sec:subject}

Кратко рассмотрим предметную область, задачи которой подлежат автоматизации в ходе выполнения курсового проекта.

Рассматриваемая предметная область – мобильное приложения для управления ``умной'' гирляндой. Основная задача предметной области – управление световым оборудованием в различных условиях.

Приложение для управления световым оборудованием (``умной'' гирляндой) представляет собой приложение, которое будет поставляться вместе с самим световым оборудованием. С помощью этого приложения, пользователь сможет управлять отображением различных анимаций на гирлянде (как заранее созданных, так и тех, что пользователь создал сам), калибровать приложение на основе текущего расположения гирлянды, чтобы большинство анимаций правильно отображались на ней. Также пользователь будет иметь возможность связывать несколько гирлянд в одну сеть, чтобы назначать им определенную последовательность из анимаций (проигрывать анимацию последовательно, от первой гирлянды к последней, либо отображать анимацию одновременно). Пользователь будет иметь гибкие возможности настройки каждой гирлянды (управление видимостью wi-fi модуля, имя wi-fi точки). Также будет реализована возможность пересылать самостоятельно созданные анимации другим людям, которые будут зарегистрированы в приложении.

\subsection{Индустрия электронного светового оборудования}
\label{sec:subject:industry}

Электронное световое оборудование~--- это последовательность из сопряженных световых элементов, например светодиодов, под управлением какого-либо электронного устройства.

Начиная с конца 19-го века, когда русский электротехник Павел Николаевич Яблочков изобрел первую электрическую лампочку, и по сегодняшний день, индустрия электронного светового оборудования постоянно увеличивается в размерах. Разнообразие товаров растет, на рынке присутствует большой спрос, но и такая же большая конкуренция. Множество компаний получают контракты на освещение больших архитектурных сооружений, поставку светового оборудования для проведения различных музыкальных концертов (светомузыка) и так далее. Сама индустрия дистрибуции электронного светового оборудования ориентируется как на рынок физических лиц (украшения на елку, комнатное освещение, внешние украшения для дома), так и на юридических лиц (внешнее и внутренне освещение офисов, светомузыка, световые скульптуры). 

Рынок домашнего освещения становится все более и более популярным, особенно зарубежом. В США и Европе различные украшения на дом в Рождество очень популярны. В нашей стране схожее тоже можно наблюдать в Рождество и Новый год, но там масштабы и разнообразия продуктов куда шире. Люди тратят много денег, чтобы приобрести целые световые комплексы, позволяющие им уникально украсить свой дом и отличаться от соседей (Рисунок~\ref{fig:subject:industry:example}).

~
\begin{figure}[H]
\centering
	\includegraphics[scale=0.35]{figures/home_lightings.jpg}
	\caption{Пример внешнего домового освещения}
	\label{fig:subject:industry:example}
\end{figure}

Так как рынок постоянно растет, сами товары совершенствуются и усложняются, то встает вопрос об управлении этими вещами. Контроллеров с маленькими дисплеями, либо просто кнопок переключения между анимациями теперь не хватает. Покупателям нужно куда больше функционала, плюс к тому, хорошо было бы иметь возможность постоянной поддержки системы управления, будь то ``патчи'' с исправлениями ошибок, либо добавление нового функционала. Для этого отлично подойдут устройства, которые и так у нас постоянно под рукой~--- мобильные телефоны. Сама же технология управления различными техническими устройствами с помощью мобильного телефона уже обзавелась названием ``Интернет вещей'' (англ. Internet of Things) или же коротко~-- IoT.

\input{content/sections/01_02_mobileApps}

\input{content/sections/01_03_iot}
