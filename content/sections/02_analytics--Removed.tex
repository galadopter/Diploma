\section{Анализ требований к программному средству и разработка функциональных требований}
\label{sec:domain}

\subsection{Функциональная модель программного средства}
\label{sec:domain:model}

Функциональная модель программного средства представлена в виде\linebreakсхемы алгоритма управления гирляндой и диаграммы вариантов использования и информационной модели предметной области. Варианты использования отражают функциональность системы в ответ на внешние воздействия с точки зрения получения значимого результата для пользователей. Информационная модель предметной области в дальнейшем будет использоваться при проектировании базы данных для программного средства.

\subsubsection{ } Варианты использования программного средства
\label{sec:domain:model:use_cases}

Перед началом проектирования необходимо проанализировать как и зачем могут использоваться гирлянды. Данный анализ будет проводится с точки зрения как рядового пользователя, который покупает гирлянду для развлечения, так и с точки зрения предпринимателя, который использует гирлянду для рекламы.

Для обычных пользователей, которые приобрели программно-аппаратный продукт, наиболее важными функциями являются обилие способов для изменения анимаций (изменение скорости, яркости, цветов), возможность обмениваться анимациями с друзьями, а также возможность откалибровать гирлянду.

Для предпринимателей, помимо выше перечисленных функций, будет также нужна возможность писать текст на гирлянде, чтобы привлекать покупателей и сообщать им свои контакты.

По результатам анализа предметной области и существующих аналогов можно сделать вывод, что проектируемое программное-аппаратное средство должно поддерживать ряд функций которые позволят манипулировать эффектами на гирлянде, ключевыми из которых являются следующие:

\begin{itemize}
	\item \emph{Подключение}. Возможность подключиться к гирлянде через вай- фай.
	\item \emph{Создание анимаций}. Создание собственных анимаций необходимо для того, чтобы количество анимаций было безграничным. Так же это важно для того, чтобы пользователи могли делиться созданными анимациями с другими пользователями.
	\item \emph{Два типа гирлянд/анимаций: Net и String}. Net: гирлянда состоящая из 200 лампочек и которые связаны между собой так, что образуют прямоугольник. String: гирлянда состоящая из 100 лампочек, которые связаны друг с другом последовательно и образуют линию.
	\item \emph{Редактирование анимаций}. Возможность редактировать анимации без необходимости создавать новые. Изменять можно скорость воспроизведения, цвет, а также интенсивность.
	\item \emph{Удаление анимаций}. Возможность удалять созданные анимации.
	\item \emph{Калибровка гирлянды при помощи камеры}. Удобство использования значительно повышается, когда предоставляется возможность развесить гирлянду произвольным образом, а затем откалибровать, чтобы все анимации отображались корректно.
	\item \emph{Трансформация анимации в текст}. Возможность создать анимацию из текста (будет отображаться только на гирляндах типа Net в виде бегущей строки).
	\item \emph{Сообщения} между пользователями позволяют обмениваться анимациями.
	\item \emph{Обновления прошивки гирлянды} для того, чтобы добавлять новые функции в приложение, либо поменять старые, необходимо.
\end{itemize}

Диаграмма вариантов использования, разработанная с использованием нотации \uml, представлена на рисунке%~\ref{fig:domain:model:use_cases:model}.


% \begin{figure}[H]
% \centering
% 	\includegraphics[scale=0.3]{Drawings/useCases.png}
% 	\caption{Диаграмма вариантов использования ПС}
% 	\label{fig:domain:model:use_cases:model}
% \end{figure}

\pagebreak
Рассмотрим подробно представленные на рисунке прецеденты.

\emph{Регистрация, аутентификация и авторизация} -- функции, которые доступны для роли <<Гость>> (пользователь, не зарегистрированный в системе). В первой версии приложения планируется реализация собственной системы авторизации; в дальнейшем будет добавлена возможность регистрации с помощью внешних поставщиков данных (Google, VK, Facebook и др.).

Для незарегистрированного пользователя также будет предоставляться доступ к основному функционалу приложения. Однако, такие возможности, как, например, обмен сообщениями, получения обновлений прошивки гирлянды, будут доступны только для авторизованных пользователей. 

После регистрации пользователь получает доступ к \emph{обновлению прошивок}, системе \emph{сообщений}. Однако, для регистрации необходимо предоставить свой телефон, на который придет код активации, чтобы идентифицировать данного пользователя. Только после подтверждения телефона пользователи получают доступ к отправке анимаций, которые они создали. 

Для \emph{калибровки гирлянды} планируется использовать камеру телефона пользователя. Для калибровки будет достаточно выбрать соответствующее меню в приложении, навести камеру на разложенную гирлянду, и начать процесс калибровки.  

Возможность \emph{обновлять прошивку} на гирлянде предусматривается только для зарегистрированных пользователей. Проверка на наличие обновлений осуществляется каждый раз, когда пользователь заходит в приложение с подключенным интернетом.

Одной из самых главных функций, разделяемых всеми подтвержденными пользователями, является \emph{просмотр созданных анимаций}. Представление созданных анимаций в виде таблицы является наиболее удобным способом. Каждая ячейка таблицы отображает выбранный тип гирлянды, на котором проигрывается анимация. 

Функция коммуникации реализуется с помощью \emph{обмена сообщениями}. Она реализуется при помощи возможности отправки различных созданных анимаций, в том числе и текстовых анимаций. Кроме того, она способствует большей вовлеченности пользователей в творческий процесс создания анимаций.

Основная вид деятельности пользователя -- редактирование анимаций. Для этого проектируются следующие функции:

\begin{itemize}
	\item \emph{изменение скорости}, что подразумевает увеличение или уменьшение скорости воспроизведения анимации;
	\item \emph{изменение цветов} дает возможность менять цвета у анимации. Для каждой анимации должен быть свой набор изменяемых цветов;
	\item \emph{изменение интенсивности} для каждой анимации проявляется различным образом: изменяет плотность горящих лампочек.
\end{itemize}

Для привлечения внимания к своей продукции можно использовать различные анимированные надписи. Тогда представляют интерес функция \emph{трансформации текста в анимацию}. Также данная функция вместе с функцией \emph{сообщений} позволяет пользователям поздравлять друг друга с праздниками.

Программное обеспечение должно поддерживать различные типы гирлянд. В первой версии программного обеспечения будет поддержка гирлянд двух типов: Net и String

\subsubsection{ } Разработка инфологической модели базы данных 
\label{sec:domain:model:db}

Исходя из необходимости использования в проектируемом приложении базы данных, разработаем ее инфологическую модель. Для ее создания будем использовать расширение диаграммы классов \uml, предназначенное для моделирования баз данных.

Одной из особенностей разработанной модели является отсутствие в ней типов перечисления. Вместо них повсеместно (например, в таблицах Animation, User, TextAnimation, NetAnimation и некоторых других) используется целочисленный тип Byte. Это связано с особенностями планируемой к использованию СУБД. По сути, это заглушки, вместо которых будут использоваться соответствующие типы перечислений.

Можно заметить, что все остальные используемые типы данных соответствуют типам языка программирования Swift. Причиной использования данной нотации в модели является то, что для доступа к базе данных будет использоваться клиентская библиотека, реализованная для платформы iOS. Кроме того, данная библиотека является очень важным ограничением, которое влияет на выбор архитектуры и используемые технологии.

В информационной модели предусмотрены возможности для обеспечения безопасности аутентификации и авторизации: в таблице User предусмотрены поля для хранения хешированного пароля и соли, и, вдобавок, строки с названием хеширующего алгоритма.

В информационной модели предусмотрены возможности для обеспечения безопасности аутентификации и авторизации: в таблице User предусмотрены поля для хранения телефона.

Предполагается, что в проектируемой системе не будет возможности изменения набора ролей -- существующие выбранные роли очень жестко привязываются к структуре БД. С другой стороны, предметная область и не предполагает выделения большого количества ролей, а практически любое действующее лицо можно свести к одной из существующих.

В проектируемой системе предполагается существование иерархии сущностей. На вершине находятся анимации; предполагается поддержка многих типов гирлянд одной системой. Гирлянды могут иметь разное количество лампочек и разное количество поддерживаемых цветов.

\subsection{Разработка спецификации функциональных требований}
\label{sec:domain:specification}

С учетом требований, определенных в подразделе \ref{sec:analysis:specification}, представим детализацию функций проектируемого ПС.

\subsubsection{ } Функция регистрации 
\label{sec:domain:specification:signup}


Функция регистрации должна быть реализована с учетом следующих требований:

\begin{enumerate}
	\item процесс регистрации инициируется пользователем системы (на рисунке~\ref{fig:domain:model:use_cases:model} представлен в виде роли <<Гость>>);
	\item функция реализуется при помощи Google Messaging, которая позволяет производить аутентификацию при помощи мобильного телефона, а также Google Authentification, которая позволяет зарегистрироваться при помощи электронной почты и пароля;
	\item для регистрации пользователь обязан ввести мобильный телефон или электронную почту и пароль;
	\item правильность предоставленного адреса электронной почты должна проверяться путем отправки письма со ссылкой, переход по которой означает подтверждение пользователя. Правильность введенного телефона должна проверяться путем отправки смс с кодом, который необходимо ввести в поле для проверки;
	\item хранение пароля допускается только в хешированном виде; применяющийся алгоритм должен по криптостойкости быть равным или превосходить алгоритмы семейства SHA-2. Использованиеи обязательно;
	\item должна быть предусмотрена возможность смены пароля и электронной почты после регистрации. Правильность нового адреса почты проверяется отправкой на него письма с подтверждающей ссылкой.
\end{enumerate}

\subsubsection{ } Функция аутентификации 
\label{sec:domain:specification:authentication}

Функция аутентификации должна быть реализована с учетом следующих требований:

\begin{enumerate}
	\item Инициатором является пользователь, при этом ему необходимо предоставить адрес электронной почты и пароль, заданные при регистрации;
	\item Должна быть реализована возможность повторной аутентификации пользователя без необходимости ввода какой-либо информации;
	\item Должна быть реализована возможность восстановления пароля:
	\begin{enumerate}
		\item Для восстановления пароля пользователь должен предоставить адрес электронной почты, зарегистрированный в системе;
		\item На предоставленный адрес высылается уникальная ссылка;
		\item После перехода пользователем по данной ссылке ему предоставляется возможность установить новый пароль.
	\end{enumerate}
\end{enumerate}

\subsubsection{ } Калибровка гирлянды
\label{sec:domain:specification:roles}

При реализации калибровки следует учесть требования:

\begin{enumerate}
	\item калибровка должна отрабатывать не более чем за 10 секунд;
	\item пользователь должен держать телефон перед началом обработки данных не более 3 секунд;
    \item процесс калибровки должен запускаться сразу же, после того как пользователь навел камеру на гирлянду;
	\item процент распознанных лампочек должен быть не менее 75\%;
	\item процент правильно распознанных лампочек должен быть не менее 80\%.
\end{enumerate}

\subsubsection{} Функция создания анимаций
\label{sec:domain:specification:messages}

Функция создания анимаций должна реализовывать следующие требования:

\begin{enumerate}
	\item должна быть возможность создавать анимации для разных типов гирлянд;
	\item должна быть возможность выбрать размер кисти для рисования;
	\item если устройство подключено к гирлянде, анимации должны отображаться на гирлянде;
	\item должна быть возможность выбрать цвет каждой лампочки;
	\item анимации должны быть разделены на статические и динамические.
\end{enumerate}

\subsubsection{} Функция просмотра анимаций
\label{sec:domain:specification:agenda}

Просмотр анимаций входит в число основных функций разрабатываемого приложения. При реализации данной функции необходимо учесть следующие требования:

\begin{enumerate}
	\item необходимо обеспечить отображение анимаций в виде последовательного списка, отсортированного в порядке возрастания даты создания;
	\item каждая ячейка в списке должна проигрывать анимацию;
	\item необходимо обеспечить возможность фильтровать гирлянды по типам Net и String. После применения фильтра, ячейки отображают лампочки так, как они расположены на выбранном типе;
    \item должна быть реализована возможность просматривать анимации которые были получены от других пользователей.
\end{enumerate}

\subsubsection{} Функция редактирования анимаций
\label{sec:domain:specification:subjects}

Функция редактирования анимаций должна быть реализована с учетом следующих требований:

\begin{enumerate}
	\item должна быть возможность редактировать как стандартные анимации, так и те, которые созданы пользователями;
	\item редактирование анимаций должна происходить на отдельном экране;
	\item должна быть возможность изменять скорость, цвет, интенсивность анимации;
	\item изменения должны применяться сразу же;
	\item должна быть возможность сохранить или отменить изменения.
\end{enumerate}

\subsubsection{} Функция отправки анимаций
\label{sec:domain:specification:tasks}

Функция отправки анимаций также является ключевой. При ее реализации должны быть учтены следующие требования:

\begin{enumerate}
	\item у пользователя должна быть возможность отправлять любую созданную анимацию;
	\item пользователь может отправить анимацию любому зарегистрированному пользователю в системе, телефон которого есть у него в контакте, либо введя телефон самостоятельно;
	\item должна быть возможность выбрать кому будет отправлена анимация из отфильтрованного списка контактов;
	\item получатель анимации должен увидеть пуш-уведомление если в данный момент он не пользуется приложением;
    \item в пуш-уведомлении должно отображаться имя отправителя и название отправленной анимации;
    \item у получателя должна быть возможность как сохранить полученную анимацию, так и удалить ее; 
    \item все полученные анимации отображаются на отдельном экране.
\end{enumerate}



\subsubsection{} Обновление прошивки гирлянды
\label{sec:domain:specification:student_history}

Следующая группа требований относится к обновлению прошивки:

\begin{enumerate}
	\item должна быть возможность загрузить любую прошивку гирлянды из имеющихся;
	\item по умолчанию должна быть реализована функция обновления прошивки каждый раз, когда пользователь подключается к сети интернет. 
\end{enumerate}
