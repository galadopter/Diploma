\subsection{Аналитический обзор литературных источников}
\label{sec:analysis:literature}

Далее приводится анализ сведений, которые влияют на формулирование требований, выбор архитектуры и дальнейшее проектирование и разработку программного средства.

\subsubsection{} Кроссплатформенность приложений
\label{sec:analysis:literature:crossplatform}

Как несколько десятилетий назад, так и в настоящее время выбор платформы является серьезным ограничением для всех последующих этапов разработки. Однако уже начали появляться технологии, которые позволяют использовать однажды написанный код на многих платформах. Сегодня все больше приложений создается сразу для нескольких платформ, а приложения, созданные изначально для одной платформы, активно адаптируются под другие. %\cite{habr_crossplatform}. 
Разработчик, изучающий какую-либо из таких технологий, получает конкурентное преимущество, поскольку за счет расширения количества платформ расширяется круг задач, над которыми он может работать. Поэтому кроссплатформенность -- реальная или потенциальная -- является одним из факторов, который необходимо учитывать при выборе технологий реализации проекта.

\subsubsection{} Обзор целевых платформ
\label{sec:analysis:literature:platforms}

Несмотря на планируемое использование кроссплатформенных технологий, поддержка всех платформ может затребовать значительно больших сред\-ств и времени, чем есть в наличии для выполнения дипломного проектирования. Поэтому, необходимо осуществить выбор одной основной платформы с расчетом на продолжение разработки и реализацию проекта для других платформ. Рассмотрим достоинства и недостатки основных.

\emph{Настольное приложение} – программное средство, которое запускает- ся локально на компьютере пользователя. При его создании появляется возможность использования всех преимуществ аппаратного обеспечения, которым оснащен компьютер, например: прямой доступ к видеокарте, внешним устройствам. Кроме того, появляется возможность взаимодействия с другими установленными приложениями.

Тем не менее у настольных приложений есть и ряд недостатков. Когда пользователь работает удаленно, возникают проблемы, связанные с сетью, соединениями, сетевыми экранами. Один из самых больших недостатков заключается в огромной сложности развертывания приложения на десятки или сотни машин, их конфигурирования, периодического обновления \cite{msdn_desktop_vs_web}.

\emph{Веб-приложения}, в отличие от настольных, работают на удаленном аппаратном обеспечении и поставляются пользователю через браузер \cite{web_based_vs_desktop}. При их использовании разработчик избавляется от необходимости поддерживать установку большого числа зависимостей, он всегда может быть уверен, что все пользователи используют самую последнюю версию приложения. Вычислительные операции могут производиться на мощном сервере, а результаты вычислений поставляться пользователю -- так называемая концепция <<тонкого клиента>> \cite{desktop_vs_web_deeper_look}.

Несмотря на то, что ресурсоемкие вычисления производятся на сервере, задержки при передаче, особенно при нестабильном соединении, сама потребность в постоянном интернет соединении могут значительно снизить удобство пользования приложением. Кроме того, размер загружаемого при каждом запуске кода и ресурсов может значительно увеличить траты пользователя, особенно если он использует дорогое мобильное подключение к интернету.

\emph{Мобильное приложение} - программное обеспечение, предназначенное для работы на смартфонах, планшетах и других мобильных устройствах. Многие мобильные приложения предустановлены на самом устройстве или могут быть загружены на него из онлайновых магазинов приложений, таких как App Store, BlackBerry App World, Google Play, 1mobile market, Windows Phone Store, Яндекс.store и других, бесплатно или за плату.

Для \emph{мобильных приложений} актуальны ограничения платформ, на которых они запускаются, такие как меньшие размеры экранов, более медленные процессоры, ограниченное энергопотребление. Несмотря на это, мобильные устройства часто находятся рядом с пользователями, появляется возможность использования мгновенных оповещений \cite{desktop_mobile_differences}. Вместе с этим, большинство смартфонов оснащено модулями, такими как GPS, камера, NFC, что пре\-дос\-тав\-ля\-ет разработчику новые возможности по их использованию.

На основании рассмотренных характеристик различных платформ можно осуществить выбор одной из них, которая и станет целевой для разработки.

\subsubsection{} Обзор архитектурных стилей
\label{sec:analysis:literature:architecture}

Далее необходимо рассмотреть применяющиеся на практике архитектурные стили, провести их анализ и по результатам осуществить выбор архитектуры, которая затем будет применяться при проектировании программного средства.

Под разработкой \emph{архитектуры} понимают специфицирование структуры всей системы: глобальную организацию и структуру управления, протоколы коммуникации, синхронизации и доступа к данным, распределение функциональности между компонентами системы, физическое размещение, состав системы, масштабируемость и производительность \cite{introduction_to_architecture}. Набор принципов, используемых в архитектуре, формирует \emph{архитектурный стиль}. Применение архитектурных стилей упрощают решение целого класса абстрактных проблем \cite{architecture_volosevich}.

При проектировании архитектуры программной системы почти никогда не ограничиваются единственным архитектурным стилем, поскольку они могут предлагать решение каких-либо проблем в различных областях. В таблице~\ref{table:analysis:architectures:categorization} приведен вариант категоризации архитектурных стилей \cite{application_architecture_guide}.


\begin{table}[ht]
\caption{Категоризация архитектурных стилей}
\label{table:analysis:architectures:categorization}
  \begin{tabular}{|>{\raggedright}m{0.27\textwidth} 
                  |>{\raggedright\arraybackslash}m{0.675\textwidth}|}
  \hline {\begin{center} Категория \end{center} } &{ \begin{center} Архитектурный стиль \end{center} } \\
  \hline Связь & SOA (Service-oriented architecture -- архитектура, ориентированная на сервисы), Шина сообщений \\
  \hline Развертывание & Клиент-серверный, трехуровневый, N-уровневый \\
  \hline Предметная область & DDD (Domain-driven design -- проблемно-ориентированное проектирование) \\
  \hline Структура & Компонентный, объектно-ориентированный, многоуровневый\\
  \hline
  \end{tabular}
\end{table}

\emph{Сервис-ориентированная архитектура} позволяет приложениям предоставлять некоторую функциональность с помощью набора слабосвязанных автономных сервисов; связь между сервисами обеспечивается с помощью заранее определенных контрактов. Данный стиль предоставляет следующие преимущества \cite{application_architecture_guide}:

\begin{itemize}
	\item повторное использование сервисов снижает стоимость разработки;
	\item автономность и использование формальных контрактов способствует слабой связанности и повышает уровень абстракции;
	\item сервисы могут использовать возможность автоматического обнаружения и определения интерфейса;
	\item сервисы и использующие их приложения могут быть развернуты на различных платформах.
\end{itemize}

Архитектура \emph{шины сообщений} описывает принципы построения систем, которые используют обмен сообщениями как способ связи. Наиболее часто при реализации данной архитектуры используется модель маршрутизатора сообщений или шаблон издатель-подписчик. Главные преимущества использования данного архитектурного стиля \cite{application_architecture_guide}:

\begin{itemize}
	\item расширяемость, которая заключается в возможности добавлять и удалять приложения без влияния на другие;
	\item снижается сложность приложений, так как единственный интерфейс, который они должны поддерживать -- интерфейс общей шины;
	\item гибкость, которая заключается в возможности подстраиваться под биз\-нес-требования или желания пользователей через изменения конфигурации или параметров маршрутизации сообщений;
	\item слабая связанность, поскольку единственное, чем связаны приложения -- интерфейс общей шины;
	\item масштабируемость, которая заключается в возможности в случае необходимости присоединения к шине нескольких экземпляров одного и того же приложения.
\end{itemize}

\emph{Клиент-серверная архитектура} описывает распределенную систему, которая включает независимые системы сервера, клиента и соединяющую их сеть. Иногда данную архитектуру называют двухзвенной. Из преимуществ выделяют следующие \cite{architecture_volosevich}:

\begin{itemize}
	\item безопасность: все данные хранятся на сервере, обеспечивающем больший уровень безопасности, чем отдельные клиенты;
	\item централизованный доступ к данным, который предоставляет возможность более легкого управления, чем в других архитектурах;
	\item устойчивость и легкость поддержки: роль сервера могут выполнять несколько физических компьютеров, объединённых в сеть; благодаря этому клиент не замечает сбоев или замены отдельного серверного компьютера.
\end{itemize}

\emph{Многоуровневая архитектурный стиль} заключается в группировании схожей функциональности приложения по уровням, которые выстроены в вертикальную структуру. Уровни связаны слабо, связь между ними осуществляется по явно установленным протоколам. Строгий вариант архитектуры предполагает, что компоненты какого-либо уровня могут взаимодействовать только с компонентами одного нижележащего уровня; ослабленный вариант разрешает взаимодействие с компонентами любого из нижележащих уровней. Использование данного архитектурного стиля предлагает следующие преимущества \cite{application_architecture_guide}:

\begin{itemize}
	\item есть возможность осуществлять изменения на уровне абстракций;
	\item изолированность: изменения на каких-либо уровнях не влияют на другие, что снижает риск и минимизирует воздействие на всю систему;
	\item разделение функциональности помогает управлять зависимостями, что приводит к большей управляемости всего кода;
	\item независимые уровни предоставляют возможность повторного использования компонентов;
	\item строго-определенные интерфейсы способствуют повышению тестируемости компонентов.
\end{itemize}

\emph{Многозвенная архитектура} предлагает схожее с многоуровневой архитектурой разбиение функциональности, отличие же заключается в предлагаемом размещении звеньев на физически обособленной машине \cite{architecture_volosevich}. Предлагается использовать данный стиль в случаях, когда компоненты одного звена могут проводить дорогие в ресурсном отношении вычисления, так что это может сказаться на других уровнях, или когда некоторую чувствительную информацию переносят со звеньев уровня представления на уровень бизнес-логики приложения. Далее приведены главные преимущества использования многозвенной архитектуры:

\begin{itemize}
	\item удобство сопровождения, которое заключается в возможности внесения изменений и обновлений в некоторые компоненты при минимальном влиянии на всё приложение;
	\item масштабируемость, которая возникает из распределенного развертывания;
	\item гибкость, которая появляется благодаря предыдущим двум пунктам;
	\item масштабируемость также приводит к повышению доступности приложения.
\end{itemize}

\emph{Проблемно-ориентированный архитектурный стиль} основывается на предметной области, ее элементах, их поведении и связях между ними. Для применениях данного стиля необходимо иметь хорошее понимание предметной области или людей, которые бы смогли объяснить ее специфику разработчикам. Первое, что нужно сделать при принятии данного стиля -- выработать единый язык, который бы знала вся команда разработчиков, который бы был избавлен от технических жаргонизмов и содержал только термины предметной области -- только так можно избежать проблемы непонимания между участникам \cite{ddd_quickly}. Преимущества, которые может предложить данный стиль:

\begin{itemize}
	\item упрощение коммуникации между участниками процесса разработки благодаря выработке единого языка;
	\item модель предметной области обычно является расширяемой и гибкой при изменениях условий и бизнес-требований;
	\item хорошая тестируемость.
\end{itemize}


\pagebreak
\emph{Компонентная архитектура} основывается на декомпозиции системы в отдельные функциональные или логические компоненты, которые раскрывают другим компонентам только заранее определенные интерфейсы \cite{application_architecture_guide}. Преимущества данного подхода:

\begin{itemize}
	\item легкость развертывания, которая заключается в обновлении компонентов без влияния на другие;
	\item использование компонентов сторонних разработчиков позволяет снизить расходы на разработку и поддержку;
	\item поддержка компонентами заранее определенных интерфейсов позволяет упростить разработку;
	\item возможность переиспользования компонентов также оказывает влияние на снижение стоимости.
\end{itemize}

\emph{Объектно-ориентированный архитектурный стиль} выражается в разделении функциональности системы на множество автономных объектов, каждый из которых содержит некоторые данные и набор методов поведения, свойственных объекту. Обычной практикой является определение классов, соответствующих объектам предметной области. Применение данного стиля предоставляет следующие преимущества \cite{application_architecture_guide}:

\begin{itemize}
	\item соотнесение классов программы и объектов реального мира делает программное средство более понятным;
	\item полиморфизм и принцип абстракции предоставляет возможность повторного их использования;
	\item инкапсуляция повышает тестируемость объектов;
	\item улучшение расширяемости, которое возникает благодаря тому, что изменения в представлении данных не оказывают влияния на внешний интерфейс объекта, что не ограничивает его способность взаимодействовать с другими объектами;
	\item высокая сцепленность объектов, которая достигается использованием разных объектов для разного набора действий.
\end{itemize}

Таким образом, применение рассмотренных архитектур на этапе проектирования окажет большое влияние на успешность всего процесса разработки; какие бы стили не были применены, можно быть уверенным в правильности выбора за счет того, что у всех из них есть определенные и зачастую разные преимущества.

\newpage
\subsubsection{} Проектирование баз данных
\label{sec:analysis:literature:db}

В настоящее время сложно представить сложные приложения, которые бы не использовали специальные средства для хранения информации.

База данных -- представленная в объективной форме совокупность самостоятельных материалов, систематизированных таким образом, чтобы эти материалы могли быть найдены и обработаны с помощью ЭВМ. Модель базы данных -- описание базы данных с помощью  определенного (в т.ч. графического) языка на некотором уровне абстракции.
%
Основные задачи проектирования баз данных:
\begin{itemize}
	\item обеспечение хранения в БД всей необходимой информации;
	\item обеспечение возможности получения данных по всем необходимым запросам;
	\item сокращение избыточности и дублирования данных;
	\item обеспечение целостности базы данных.
\end{itemize}

Выделяют следующие уровни моделирования \cite{kulikov_db_workbook}:
\begin{itemize}
	\item инфологический уровень: описание предметной области без привязок к каким-либо средствам реализации: языкам программирования, СУБД, и т.д;
	\item даталогический уровень: модель предметной области в привязке к сре\-д\-с\-т\-вам реализации;
	\item физический уровень: описывает конкретные таблицы, связи, индексы, методы хранения, настройки производительности, безопасности и т.д.
\end{itemize}

При ошибках моделирования могут возникать аномалии операций с БД. Аномалия -- противоречие между моделью предметной области и моделью данных, поддерживаемой средствами конкретной СУБД~\cite{kulikov_db_workbook}.

Выделяют следующие виды аномалий:
\begin{itemize}
	\item Аномалия вставки: при добавлении данных, часть которых у нас отсутствует, мы вынуждены или не выполнять добавление или подставлять пустые или фиктивные данные.
	\item Аномалия обновления – при обновлении данных мы вынуждены обновлять много строк и рискуем часть строк «забыть обновить».
	\item Аномалия удаления – при удалении части данных мы теряем другую часть, которую не надо было удалять.
\end{itemize}

Для устранения аномалий существует процесс нормализации БД.

Нормализация -- группировка и/или распределение атрибутов по отношениям с целью устранения аномалий операций с БД, обеспечения целостности данных и оптимизации модели БД.

Отношение находится в первой нормальной форме, если все его атрибуты являются атомарными. Атрибут считается атомарным, если в предметной области не существует операции, для выполнения которой понадобилось бы извлечь часть атрибута.

Отношение находится во второй нормальной форме, если оно находится в первой нормальной форме, и при этом  любой атрибут, не входящий в состав ПК, функционально полно зависит от ПК.

Отношение находится в третьей нормальной форме, если оно находится во второй нормальной форме, и при этом любой его неключевой атрибут нетранзитивно зависит от первичного ключа~\cite{kulikov_db_workbook}.

Как правило, третья нормальная формы принимается достаточной и, поскольку дальнейшая нормализация может породить избыточность, то она обычно не проводится.
