\subsection{Информационная модель системы}
\label{sec:develop:umlDiagrams}

При проектировании системы для поиска, вызова и оплаты такси были выделены такие сущности, как:
\begin{itemize}
	\item Анимация
	\item Лампочка
	\item Отображение
	\item Сеть
	\item Устройство (гирлянда)
	\item Пользователь
\end{itemize}

Представление информационной модели в графическом виде приведено на диаграмме Б.1 приложения Б. 

Сущность Анимация необходима для отображения стандартных и пользовательских анимаций в приложении и содержит в себе следующие атрибуты:
\begin{itemize}
	\item name~-- имя анимации
	\item id~-- id анимации, используется при отправке анимации на гирлянду (0~-- посылается через байты, 1, 2,...~-- запускает анимацию внутри самой гирлянды)
	\item previewType~-- тип анимации (default, custom)
	\item totalTime~-- полное время анимации
	\item speed~-- скорость анимации
	\item type~-- тип отображения в приложении (string, tree, custom, grid)
	\item createdDate~-- дата создания
	\item currentColorIndex~-- выбранный набор цветов
	\item maximumColorsCount~-- максимальное количество цветов в наборе
	\item maximumSpeed~-- максимальное значение скорости
	\item minimumSpeed~-- минимальное значение скорости
	\item intensity~-- интенсивность анимации
	\item lamps~-- лампочки анимации
	\item editOptions~-- доступные опции для редактирования
	\item colors~-- наборы цветов
\end{itemize}