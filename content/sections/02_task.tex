\sectioncentered*{Описание предметной области}
\addcontentsline{toc}{section}{Описание предметной области}
\label{sec:analysis}

Кратко рассмотрим предметную область, задачи которой подлежат автоматизации в ходе выполнения курсового проекта.

Рассматриваемая предметная область – мобильное приложения для управления ``умной'' гирляндой. Основная задача предметной области – поиск, оплата и вызов такси.

Конечный успех программного проекта во многом определяется до начала конструирования: на этапе подготовки, которая проводится с учетом всех особенностей проекта.

Первое предварительное условие, которое нужно выполнить перед конструированием, -- ясное формулирование проблемы, которую система должна решать. Общая цель подготовки — снижение риска: адекватное планирование позволяет исключить главные аспекты риска на самых ранних стадиях работы, чтобы основную часть проекта можно было выполнить максимально эффективно. 

Главный факторы риска в создании ПО — неудачная выработка требований. Требования подробно описывают, что должна делать программная система. Внимание к требованиям помогает свести к минимуму изменения системы после начала разработки \cite{code_complete}.

Перед формулированием требований необходимо изучить ряд вопросов, которые напрямую влияют на все дальнейшие этапы разработки. В частности, необходимо рассмотреть вопросы выбора платформ, архитектуры. По результатам анализа можно будет составить техническое задание к проектируемому программному средству, которое станет основой для составления функциональных требований.

%\input{sections/01_01_sources}

%\input{sections/01_02_analogues}

%\input{sections/01_03_specification}
