\sectioncentered*{Введение}
\addcontentsline{toc}{section}{Введение}
\label{sec:introduction}

Каждый новый год во многих уголках планеты люди начинают украшать свои дома и магазины различными новогодними игрушками. В том числе и гирляндами. И перед каждым человеком стоит вопрос: "Какие украшения мне стоит купить?". На этот вопрос сложно ответить, так как существует огромное множество различных новогодних украшений. Было бы намного удобней, если бы можно было купить одну гирлянду и настраивать ее под себя. Управление световыми эффектами представляет собой изменение цвета, скорости, создания собственных эффектов, применение предустановленных эффектов.

Перечисленные задачи особенно актуальны для предпринимателей: владельцев небольших магазинов, прилавков, а также различных торговых площадок. Так как именно они вынуждены привлекать потенциальных покупателей различными украшениями, а так же им может понадобиться изменять эффекты для того чтобы понять какие из них лучше влияют на количество покупателей. И людям для выполнения своих задач приходится комбинировать существующие гирлянды и приспосабливать их под свои нужды. Каждый человек вынужден тратить своё время на поиск украшений, тратить время на их конфигурирование. Это приводит к большим тратам, а так же возникает проблема с изменением анимаций уже после покупки.

Для задачи управления световыми эффектами реализовано не так много приложений для мобильных устройств, что благоприятно сказывается на конкурентоспособности приложения. В основном люди используют обычные гирлянды без возможности управления.

Целью настоящего курсового проекта является разработка программного средства, которое бы позволило как простым людям, так и предпринимателям, управлять световыми эффектами на гирляндах для украшения домов или привлечения внимания потенциальных покупателей. Также данный проект предоставляет возможность обмениваться созданными анимациями между пользователями. Для достижения поставленной цели необходимо решить следующие задачи:
\begin{itemize}
\item изучить предметную область управления световым оборудованием;
\item определиться с постановкой задачи, и выбрать методы для ее решения;
\item разработать модели представления системы на основе UML;
\item разработать базу данных;
\item реализовать сетевое взаимодействие с гирляндой;
\item разработать простой и удобный интерфейс приложения;
\item протестировать программное приложение.
\end{itemize}