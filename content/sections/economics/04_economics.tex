\newcommand{\investmentProfitability}{\text{Р}_\text{и}}
\newcommand{\tariffCoefficient}{\text{Т}_\text{к}}
\newcommand{\mainSalary}{\text{З}_\text{о}}
\newcommand{\mainSalaryI}{\text{З}_{\text{о}i}}
\newcommand{\subSalary}{\text{З}_\text{д}}
\newcommand{\subSalaryI}{\text{З}_{\text{д}i}}
\newcommand{\subSalaryStandard}{\text{Н}_\text{д}}

\newcommand{\ssfCharges}{\text{З}_\text{сз}\xspace}
\newcommand{\ssfChargesI}{\text{З}_{\text{сз}i}\xspace}
\newcommand{\ssfRate}{\text{Н}_\text{сз}\xspace}

\newcommand{\machineTimeCharges}{\text{Р}_\text{м}\xspace}
\newcommand{\machineTimeChargesI}{\text{Р}_{\text{м}i}\xspace}
\newcommand{\machineHourPrice}{\text{Ц}_\text{м}\xspace}
\newcommand{\hoursPerShift}{\text{Т}_\text{ч}\xspace}
\newcommand{\projectDuration}{\text{С}_\text{р}\xspace}

\newcommand{\overheadCosts}{\text{Р}_\text{н}\xspace}
\newcommand{\overheadCostsI}{\text{Р}_{\text{н}i}\xspace}
\newcommand{\overheadCostsStandard}{\text{Н}_\text{рн}\xspace}
\newcommand{\estimate}{\text{С}_\text{р}\xspace}

\newcommand{\stabilizationCosts}{\text{Р}_\text{са}\xspace}
\newcommand{\stabilizationCostsStandard}{\text{Н}_\text{рса}\xspace}

\newcommand{\totalCharges}{\text{С}_\text{п}\xspace}
\newcommand{\realizationCharges}{\text{З}_\text{рл}\xspace}
\newcommand{\realizationProfit}{\text{П}_\text{пс}\xspace}
\newcommand{\profitablenessLevel}{\text{У}_\text{рп}\xspace}

\newcommand{\priceForecast}{\text{Ц}_\text{п}\xspace}
\newcommand{\vatUnit}{\text{НДС}\xspace}
\newcommand{\vatRate}{\text{Н}_\text{дс}\xspace}
\newcommand{\priceForecastWithTaxes}{\text{Ц}_\text{о}\xspace}

\newcommand{\incomeForOne}{\text{П}_\text{ед}\xspace}
\newcommand{\yearIncome}{\text{П}_\text{г}\xspace}

\newcommand{\incomeForYear}[1]{\text{П}_\text{#1}\xspace}

\newcommand{\clearIncomeForYear}[1]{\text{ЧП}_\text{#1}\xspace}

\newcommand{\incomeTaxe}{\text{Н}_\text{приб}\xspace}

\newcommand{\discontCoefficient}{\alpha_\tau\xspace}
\newcommand{\discontCoefficientForYear}[1]{\alpha_\text{#1}\xspace}
\newcommand{\diffTimeStandard}{\text{Е}_\text{н}\xspace}
\newcommand{\effectResult}{\text{Р}_\tau\xspace}
\newcommand{\investments}{\text{З}_\tau\xspace}

\newcommand{\clearIncomeInYear}{\text{П}_{\text{ч}\tau}\xspace}
\newcommand{\averageIncome}{\text{П}_\text{чср}\xspace}

\newcommand{\BYN}{\text{руб.}\xspace}
\newcommand{\hourRateI}{\text{T}_{\text{ч}i}}
\newcommand{\dayRate}{\text{Т}_\text{д}}
\nocite{palitsyn}
\nocite{nosenko}
\section{Технико-экономическое обоснование эффективности разработки и реализации программного средства управления адресной светодиодной лентой}
\label{sec:economic}

\subsection{Характеристика программного продукта} % (fold)
\label{sec:economic:characteristic}

Главным назначением программного средства является управление адресной светодиодной лентой посредством смены анимаций, изменении настроек данных анимаций, калибровки местоположения адресной ленты в пространстве и объединения нескольких лент в сеть. В роли второстепенных функций можно выделить поддержку авторизации для последующего обмена собственными анимациями между пользователями, а также возможность превращать введённый текст в анимацию.

Данный программный продукт предназначен для свободной продажи на рынке IT компанией ООО \enquote{Сампад}, которая является резидентом ПВТ. Исходя из маркетингового исследования, лицензии на программный продукт будут востребованы на рынке в течении четырех лет: в 2019 году планируется продать 100 лицензий, в 2020 году – 200 лицензий, в 2021 году – 200 лицензий и в 2022 году – 250 лицензий.

Разработка и внедрение данной системы позволят:
\begin{itemize}
    \item улучшить качество взаимодействия пользователей с осветительными украшениями (адресной светодиодной лентой);
    \item позволит пользователям создавать собственные настройки для адресной светодиодной ленты;
    \item даст возможность пользователям делиться собственными эффектами с другими пользователями.
\end{itemize}

Экономическая целесообразность инвестиций в разработку и использование программного средства для управления адресной светодиодной лентой осуществляется на основе расчета и оценки следующих показателей:
\begin{itemize}
    \item чистая дисконтированная стоимость (ЧДД);
    \item срок окупаемости инвестиций (ТОК);
    \item рентабельности инвестиций ($\investmentProfitability$).
\end{itemize}

% subsection sec:economic:characteristic (end)

\subsection{Расчет сметы затрат и отпускной цены программного средства} % (fold)
\label{sec:economic:salaryCalculation}

Основная заработная плата исполнителей проекта определяется по следующей формуле:

\begin{equation}
    \mainSalary=\sum_{i=1}^n \hourRateI*t_i,
    \label{eq:economic:salaryCalculation:salary}
\end{equation}
\begin{explanation}
где & n & количество исполнителей, разрабатывающих программное средство; \\
    & $\hourRateI$ & часовая тарифная ставка i-го исполнителя, $\BYN$; \\
    & $t_i$ & трудоемкость работ, выполняемых i-ым исполнителем (ч).
\end{explanation}

В настоящий момент тарифная ставка 1-го разряда в компании составляет 265 рублей. Расчетная норма рабочего времени на 2019 год для пятидневной рабочей недели оставляет 168 часов. Штат исполнителей проекта состоит из руководителя проекта, инженера-программиста первой категории, дизайнера и тестировщика.

Расчет основной заработной платы исполнителей проекта представлен в таблице~\ref{table:economic:salaryCalculation:salary}.

\begin{table}[H]
\caption{Расчет основной заработной платы исполнителей}
\label{table:economic:salaryCalculation:salary}
\centering
\begin{tabular}{ |
    >{\raggedright}m{0.17\textwidth} |
    >{\centering}m{0.05\textwidth} |
    >{\centering}m{0.13\textwidth} |
    >{\centering}m{0.13\textwidth} |
    >{\centering}m{0.12\textwidth} |
    >{\centering}m{0.11\textwidth} |
    >{\centering\arraybackslash}m{0.14\textwidth} |
}

    \hline
      \multicolumn{1}{|>{\centering}p{0.17\textwidth}|}{Исполнитель}
    & \multicolumn{1}{>{\centering}p{0.05\textwidth}|}{Раз\-ряд}
    & \multicolumn{1}{>{\centering}p{0.13\textwidth}|}{Та\-риф\-ный коэффи\-циент ($\tariffCoefficient$)}
    & \multicolumn{1}{>{\centering}p{0.13\textwidth}|}{Дневная тарифная ставка ($\dayRate$), $\BYN$}
    & \multicolumn{1}{>{\centering}p{0.12\textwidth}|}{Трудоем\-кость работ ($t_i$), дней}
    & \multicolumn{1}{>{\centering}p{0.12\textwidth}|}{Премия (\%)}
    & \multicolumn{1}{>{\centering\arraybackslash}p{0.11\textwidth}|}{Заработ\-ная плата (З), $\BYN$} \\
    \hline
    \multicolumn{1}{|c|}{1} & 2 & 3 & 4 & 5 & 6 & 7 \\
    \hline
    Руководи\-тель проекта & 17 & 3,98 & 50,22 & 40 & \multirow{4}{*}{20} & 2410,74 \\
    \cline{1-5} \cline{7-7}
    Инженер-программист & 15 & 3,48 & 43,91 & 60 & & 3161,52 \\
    \cline{1-5} \cline{7-7}
    Дизайнер & 14 & 3,25 & 41,01 & 40 & & 1968,58 \\
    \cline{1-5} \cline{7-7}
    Тестировщик & 14 & 3,25 & 41,01 & 60 & & 2952,72 \\
    \hline
    \multicolumn{6}{|l|}{Основная заработная плата, $\mainSalary$} & 10493,56 \\
    \hline
\end{tabular}
\end{table}

Дополнительная заработная плата исполнителей проекта определяется по формуле:

\begin{equation}
    \subSalary=\frac{\mainSalary*\subSalaryStandard}{100},
    \label{eq:economic:salaryCalculation:subsalary}
\end{equation}
\begin{explanation}
где & $\subSalaryStandard$ & норматив дополнительной заработной платы, 10\%.
\end{explanation}
\vspace{-1em}

После подстановки значений в формулу (\ref{eq:economic:salaryCalculation:subsalary}) дополнительная заработная плата составит:
\[
    \subSalary=\frac{10493,56*10}{100}=1049,36\:\BYN
\]

Отчисления в фонд социальной защиты населения и на обязательное страхование ($\ssfCharges$) определяются в соответствии с действующими законодательными актами по формуле:

\begin{equation}
    \ssfCharges=\frac{(\mainSalary+\subSalary)*\ssfRate}{100},
    \label{eq:economic:salaryCalculation:ssfCharge}
\end{equation}
\begin{explanation}
где & $\ssfRate$ & норматив отчислений в фонд социальной защиты населения и на обязательное страхование, 34 +0,6\%.
\end{explanation}

Размер отчислений в фонд социальной защиты населения и на обязательное страхование согласно формуле (\ref{eq:economic:salaryCalculation:ssfCharge}) составит:
\[
    \ssfCharges=\frac{(10493,56+1049,36)*34,6}{100}=3993,85\:\BYN
\]

Расходы по статье \enquote{Машинное время} ($\machineTimeCharges$) определяются по формуле:

\begin{equation}
    \machineTimeCharges=\machineHourPrice*\hoursPerShift*\projectDuration,
    \label{eq:economic:salaryCalculation:machineTime}
\end{equation}
\begin{explanation}
где & $\machineHourPrice$ & цена одного машино-часа; \\
    & $\hoursPerShift$ & количество часов работы в день; \\
    & $\projectDuration$ & длительность проекта.
\end{explanation}

Стоимость машино-часа на предприятии составляет 1,5 руб. Общее время работы на машинах между 4 сотрудниками составит 200 дней. Определим по формуле (\ref{eq:economic:salaryCalculation:machineTime}) затраты по статье \enquote{Машинное время}:
\[
    \machineTimeCharges=1,5*8*200=2400\:\BYN
\]

Затраты по статье \enquote{Накладные расходы} ($\overheadCosts$) связанные с необходимостью содержания аппарата управления, вспомогательных хозяйств и опытных (экспериментальных) производств, а также с расходами на общехозяйственные нужды ($\overheadCosts$), определяются по формуле:

\begin{equation}
    \overheadCosts=\frac{\mainSalary*\overheadCostsStandard}{100},
    \label{eq:economic:salaryCalculation:overheadCosts}
\end{equation}
\begin{explanation}
где & $\overheadCostsStandard$ & норматив накланых расходов, 50\%.
\end{explanation}
\vspace{-1em}

После подстановки значений в формулу (\ref{eq:economic:salaryCalculation:overheadCosts}) накладные расходы составят:
\[
    \overheadCosts=\frac{10493,56*50}{100}=5246,78\:\BYN
\]

Общая сумма расходов по всем статьям сметы ($\estimate$) на программное обеспечение рассчитывается по формуле:

\begin{equation}
    \estimate=\mainSalaryI+\subSalaryI+\ssfChargesI+\machineTimeChargesI+\overheadCostsI,
    \label{eq:economic:salaryCalculation:estimate}
\end{equation}

По формуле (\ref{eq:economic:salaryCalculation:estimate}) получаем сумму расходов по всем статьям сметы:
\[
    \estimate=10493,56+1049,36+3993,85+2400+5246,78=23183,55\:\BYN
\]

Кроме того, организация-разработчик осуществляет затраты на сопровождение и адаптацию программного средства ($\stabilizationCosts$), которые определяются по формуле:

\begin{equation}
    \stabilizationCosts=\frac{\estimate*\stabilizationCostsStandard}{100},
    \label{eq:economic:salaryCalculation:stabilizationCosts}
\end{equation}
\begin{explanation}
где & $\stabilizationCostsStandard$ & норматив на сопровождение и адаптацию, 20\%.
\end{explanation}
\vspace{-1em}

Затраты на сопровождение и адаптацию программного продукта по формуле (\ref{eq:economic:salaryCalculation:stabilizationCosts}) составят:
\[
    \stabilizationCosts=\frac{23183,55*20}{100}=4636,71\:\BYN
\]

Общая сумма расходов на разработку (с затратами на сопровождение и адаптацию) как полная себестоимость программного средства ($\totalCharges$) определяется по формуле:

\begin{equation}
    \totalCharges=\estimate+\stabilizationCosts,
    \label{eq:economic:salaryCalculation:totalCharges}
\end{equation}

Полная себестоимость программного средства, рассчитанная по формуле (\ref{eq:economic:salaryCalculation:totalCharges}) составит:
\[
    \totalCharges=23183,55+4636,71=27820,26\:\BYN
\]

Затраты на реализацию определяются по формуле:

\begin{equation}
    \realizationCharges=\frac{\totalCharges*5}{100},
    \label{eq:economic:salaryCalculation:realizationCharges}
\end{equation}

Затраты на реализацию, рассчитанные по формуле (\ref{eq:economic:salaryCalculation:realizationCharges}) составят:
\[
    \realizationCharges=\frac{27820,26*5}{100}=1391,01\:\BYN
\]

Общие затраты на разработку и реализацию программного средства:
\[
    \text{З}=\totalCharges+\realizationCharges=27820,26+1391,01=29211,27\:\BYN
\]

Прибыль, включаемая в отпускную цену, рассчитывается по формуле:

\begin{equation}
    \realizationProfit=\frac{\totalCharges*\profitablenessLevel}{100},
    \label{eq:economic:salaryCalculation:realizationProfit}
\end{equation}
\begin{explanation}
где & $\profitablenessLevel$ & уровень рентабельности программного средства, 25\%; \\
    & $\realizationProfit$ & прибыль от реализации программного средства; \\
    & $\totalCharges$ & себестоимость программного средства.
\end{explanation}

Прибыль, включаемая в отпускную цену, составит в численном выражении:
\[
    \realizationProfit=\frac{27820,26*25}{100}=6955,07\:\BYN
\]

Так как ООО \enquote{Сампад} является резидентом ПВТ, то данная компания освобождена от налога на добавленную стоимость (НДС). Прогнозируемая отпускная цена программного средства рассчитывается по следующей формуле:

\begin{equation}
    \priceForecast=\totalCharges+\realizationProfit,
    \label{eq:economic:salaryCalculation:priceForecast}
\end{equation}

Прогнозируемая отпускная цена программного средства составит:
\[
    \priceForecast=27820,26+6955,07=34775,33\:\BYN
\]

Полученные в процессе расчетов значения представлены в таблице~\ref{table:economic:salaryCalculation:totalChargesAndPrices}.

\begin{table}[H]
\caption{Смета затрат о отпускная цена программного продукта}
\label{table:economic:salaryCalculation:totalChargesAndPrices}
\centering
\begin{tabular}{ |
    >{\raggedright}m{0.55\textwidth} |
    >{\centering}m{0.15\textwidth} |
    >{\centering\arraybackslash}m{0.22\textwidth} |
}

    \hline
    \centering Наименование статей & Усл. обозн. & Значение (руб.) \\
    \hline
    \centering 1 & 2 & 3 \\
    \hline
    Основная заработная плата испол\-ни\-те\-лей & $\mainSalary$ & 10493,56  \\
    \hline
    Дополнительная заработная плата ис\-пол\-нителей & $\subSalary$ & 1049,36  \\
    \hline
    Отчисления в фонд социальной защиты населения и фонд обязательного стра\-хо\-вания & $\ssfCharges$ & 3993,85 \\
    \hline
    Машинное время & $\machineTimeCharges$ & 2400 \\
    \hline
    Накладные расходы & $\overheadCosts$ & 5246,78 \\
    \hline
    Общая сумма расходов & $\estimate$ & 23183,55 \\
    \hline
    Расходы на освоение и сопровождение ПС & $\stabilizationCosts$ & 4636,71 \\
    \hline
    Полная себестоимость & $\totalCharges$ & 27820,26 \\
    \hline
    Затраты на реализацию & $\realizationCharges$ & 1391,01 \\
    \hline
    Прогнозируемая прибыль & $\realizationProfit$ & 6955,07 \\
    \hline
    Прогнозируемая отпускная цена & $\priceForecast$ & 34775,33 \\
    \hline
\end{tabular}
\end{table}

% subsection sec:economic:salaryCalculation (end)

\subsection{Расчет экономического эффекта от реализации программного продукта} % (fold)
\label{sec:economic:economicEffect}

На основании маркетингового исследования цена одной копии лицензии составила 170 руб. При этом лицензии на программный продует будут востребованы на рынке в течение четырех лет: в 2019 году планируется продать 100 лицензий, в 2020 году – 200 лицензий, в 2021 году – 200 лицензий и в 2022 году – 250 лицензий.

Прибыль от продажи одной лицензии программного продукта определятся по формуле:

\begin{equation}
    \incomeForOne=\text{Ц}-\frac{\totalCharges+\realizationCharges}{\text{N}},
    \label{eq:economic:economicEffect:incomeForOne}
\end{equation}
\begin{explanation}
где & $\text{Ц}$ & цена реализации одной лицензии ПО, $\BYN$; \\
    & $\text{N}$ & количество лицензий, которое будет продано с 2019 по 2022 год; \\
    & $\incomeForOne$ & прибыль, получаемая организацией-разработчиком от реализации одной копии программного продукта, руб.
\end{explanation}

Прибыль от продажи одной копии программного продукта, рассчитанная по формуле (\ref{eq:economic:economicEffect:incomeForOne}) составит:
\[
    \incomeForOne=170-\frac{27820,26+1391,01}{550}=116,89\:\BYN
\]

Экономический эффект организации-разработчика программного обеспечения в данном случае заключается в получении прибыли от его продажи множеству потребителей. Прибыль от реализации в данном случае напрямую зависит от объемов продаж, цены реализации и затрат на разработку ПО.

Суммарная годовая прибыль рассчитывается по формуле:

\begin{equation}
    \yearIncome=\incomeForOne*\text{N},
    \label{eq:economic:economicEffect:yearIncome}
\end{equation}

Суммарная годовая прибыль за каждый год, рассчитанная по формуле (\ref{eq:economic:economicEffect:yearIncome}), составит:

\begin{gather*}
    \incomeForYear{2019}=116,89*100=11689\:\BYN \\
    \incomeForYear{2020}=116,89*200=23378\:\BYN \\
    \incomeForYear{2021}=116,89*200=23378\:\BYN \\
    \incomeForYear{2022}=116,89*250=29222,50\:\BYN
\end{gather*}

Чистая прибыль рассчитывается по формуле:

\begin{equation}
    \text{ЧП}=\text{П}-\frac{\text{П}*\incomeTaxe}{100},
    \label{eq:economic:economicEffect:yearIncome}
\end{equation}
\begin{explanation}
где & $\incomeTaxe$ & ставка налога на прибыль, 12\%.
\end{explanation}
\vspace{-1em}

Таким образом, чистая прибыль за каждый год реализации программного продукта составит:

\begin{gather*}
    \clearIncomeForYear{2019}=11689-\frac{11689*12}{100}=10286,32\:\BYN \\
    \clearIncomeForYear{2020}=20572,64\:\BYN \\
    \clearIncomeForYear{2021}=20572,64\:\BYN \\
    \clearIncomeForYear{2022}=25715,80\:\BYN
\end{gather*}

% subsection sec:economic:economicEffect (end)

\subsection{Расчет показателей эффективности использования программного продукта} % (fold)
\label{sec:economic:effectData}

Для расчета показателей экономической эффективности использования программного продукта, необходимо полученные суммы результата (прироста чистой прибыли) и затрат (капитальных вложений) по годам привести к единому времени~--- расчетному году (за расчетный год принят 2019 год) путем умножения результатов и затрат за каждый код на коэффициент дисконтирования ($\discontCoefficient$), который рассчитывается по формуле:

\begin{equation}
    \discontCoefficient=\frac{1}{(1+\diffTimeStandard)^{\tau-1}},
    \label{eq:economic:effectData:discontCoefficient}
\end{equation}
\begin{explanation}
где & $\diffTimeStandard$ & норматив приведения разновременных затрат и результатов, 16\%; \\
    & $\tau$ & номер года, результаты и затраты которого приводятся к расчетному (2019-1, 2020-2, 2021-3, 2022-4).
\end{explanation}

На 01.05.2019 ставка рефинансирования ($\diffTimeStandard$) РБ составляет 10,5\%. Расчет коэффициента приведения за каждый год по формуле (\ref{eq:economic:effectData:discontCoefficient}) примут вид:

\begin{gather*}
    \discontCoefficientForYear{1}=\frac{1}{(1+0,105)^{0}}=1 \\
    \discontCoefficientForYear{2}=\frac{1}{(1+0,105)^{1}}=0,90 \\
    \discontCoefficientForYear{3}=\frac{1}{(1+0,105)^{2}}=0,82 \\
    \discontCoefficientForYear{4}=\frac{1}{(1+0,105)^{3}}=0,74
\end{gather*}

Результаты расчета показателей эффективности сведены в таблицу~\ref{table:economic:effectData:effectDataCalculation}.

Для определения экономической целесообразности инвестиций в разработку и использование необходимо определить чистый дисконтированный доход, срок окупаемости инвестиций и рентабельность инвестиций.

Чистый дисконтированный доход (ЧДД) рассчитывается по формуле:

\begin{equation}
    \text{ЧДД}=\sum_{\tau=1}^n(\effectResult*\discontCoefficient-\investments*\discontCoefficient),
    \label{eq:economic:effectData:clearDiscontIncome}
\end{equation}
\begin{explanation}
где & $n$ & расчетный период, лет; \\
    & $\effectResult$ & результат (экономический эффект), полученный в году $\tau, \BYN$; \\
    & $\discontCoefficient$ & затраты (инвестиции в разработку ПО) в году $\tau, \BYN$
\end{explanation}

\begin{table}[H]
\caption{Расчет экономического эффекта от реализации нового программного средства}
\label{table:economic:effectData:effectDataCalculation}
\centering
\begin{tabular}{ |
    >{\raggedright}m{0.26\textwidth} |
    >{\centering}m{0.08\textwidth} |
    >{\centering}m{0.13\textwidth} |
    >{\centering}m{0.13\textwidth} |
    >{\centering}m{0.13\textwidth} |
    >{\centering\arraybackslash}m{0.13\textwidth} |
}

    \hline
    \centering \multirow{2}{*}{\shortstack[c]{Наименование\\ показателей}} & \multirow{2}{*}{\shortstack[c]{Ед.\\ изм.}} & \multicolumn{4}{c|}{Расчетный период} \\
    \cline{3-6}
      &  & 2019 & 2020 & 2021 & 2022 \\
    \hline
    Экономический эффект & $\BYN$ & 5143,16 & 20572,64 & 20572,64 & 25715,80  \\
    \hline
    Коэффициент дисконтирования & $-$ & 1 & 0,90 & 0,82 & 0,74  \\
    \hline
    Дисконтированный результат & $\BYN$ & 5143,16 & 18515,38 & 16869,56 & 19029,69 \\
    \hline
    Затраты на разработку программного средства ($\totalCharges$) & $\BYN$ & 27820,26 & $-$ & $-$ & $-$ \\
    \hline
    Дисконтированные инвестиции & $\BYN$ & 27820,26 & $-$ & $-$ & $-$ \\
    \hline
    ЧДД по годам & $\BYN$ & -22677,10 & 18515,38 & 16869,56 & 19029,69 \\
    \hline
    ЧДД с нарастающим итогом & $\BYN$ & -22677,10 & -4161,72 & 12707,84 & 31737,53 \\
    \hline
\end{tabular}
\end{table}

Так как чистый дисконтированный доход больше нуля, то проект эффективен, то есть инвестиции в разработку данного ПО экономически целесообразны.

Рассчитаем рентабельность инвестиций в разработку и внедрение программного продукта ($\investmentProfitability$) по формуле:

\begin{equation}
    \investmentProfitability=\frac{\averageIncome}{\text{З}}*100,
    \label{eq:economic:effectData:investmentProfitability}
\end{equation}
\begin{explanation}
где & $\averageIncome$ & среднегодовая величина чистой прибыли за расчетный период, руб., которая определяется по формуле:
\end{explanation}

\begin{equation}
    \averageIncome=\frac{\sum_{i=1}^n(\clearIncomeInYear)}{n},
    \label{eq:economic:effectData:averageIncome}
\end{equation}
\begin{explanation}
где & $\clearIncomeInYear$ & чистая прибыль, полученная в году $\tau, \BYN$
\end{explanation}
\vspace{-1em}

По формуле (\ref{eq:economic:effectData:averageIncome}) среднегодовую величину чистой прибыли за расчетный период:
\[
    \averageIncome=\frac{5143,16+20572,64+20572,64+25715,80}{4}=14143,69\:\BYN
\]

Подставив данное значени в формулу (\ref{eq:economic:effectData:investmentProfitability}) получим рентабельность инвестиций в разработку и внедрение программного продукта:
\[
    \investmentProfitability=\frac{14143,69}{27820,26}=51\%
\]

В результате технико-экономического обоснования разработки программного средства управления адресной светодиодной лентой были получены следующие показатели:
\begin{itemize}
    \item чистый дисконтированнный доход за четыре года продаж программы составит 31737,53 руб.;
    \item затраты на разработку программного продукта окупятся на четвертый год его использования;
    \item рентабельность инвестиций составит 51\%.
\end{itemize}

Таким образом, разработка и продажа программного продукта является эффективной и инвестиции в его разработку целесообразно осуществлять.

% subsection effectData (end)