\subsection{Постановка задачи на проектирование}
\label{sec:develop:task}

Задачей курсового проекта является разработка и создание мобильного приложения для управления световым оборудованием (гирляндой). 

Основная цель разработки программного продукта – улучшение пользовательского опыта при управлении гирляндами.
Разрабатываемая система должна удовлетворять следующим требованиям:
\begin{itemize}
	\item хранение данных должно быть обеспечено с помощью мобильной базы данных Realm;
	\item должна быть реализована возможность удаленной работы с гирляндой через подключение к ней напрямую по Wi-Fi, так и через стороннюю Wi-Fi точку;
	\item должна быть реализована возможность создания и удаления пользовательских анимаций, а также редактирование стандартных анимаций (изменение скорости, частоты и набора цветов);
	\item должна быть реализована возможность отправки своих анимаций другим пользователям
	\item должна быть реализована возможность регистрации пользователей (база пользователей используется для отправки анимаций)
	\item должна быть реализована возможность объединения гирлянд в одну сеть для последующего управления
	\item должна быть реализована возможность обновления прошивки гирлянды с мобильного телефона
	\item взаимодействие мобильного приложения должно осуществляться посредством протоколов HTTP, TCP/IP и UDP.
\end{itemize}

Для авторизации пользователей, отправки анимаций и обновления прошивки должен быть реализован сторонний сервер, который будет управлять обменом данных между мобильными приложениями.
