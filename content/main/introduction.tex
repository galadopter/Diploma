\sectioncentered*{Введение}
\addcontentsline{toc}{section}{Введение}
\label{sec:introduction}

Каждый новый год во многих уголках планеты люди начинают украшать свои дома и магазины различными новогодними игрушками. В том числе и гирляндами (адресными светодиодными лентами). И перед каждым человеком стоит вопрос: \enquote{Какие украшения мне стоит купить?}. На этот вопрос сложно ответить, так как существует огромное множество различных новогодних украшений. Было бы намного удобней, если бы можно было купить одну ленту и настраивать ее под себя. Управление световыми эффектами представляет собой изменение цвета, скорости отображения, создания собственных эффектов, применение предустановленных эффектов.

Перечисленные задачи особенно актуальны для предпринимателей: владельцев небольших магазинов, прилавков, а также различных торговых площадок. Так как именно они вынуждены привлекать потенциальных покупателей различными украшениями, а так же им может понадобиться изменять эффекты для того чтобы понять какие из них лучше влияют на количество покупателей. И людям для выполнения своих задач приходится комбинировать существующие адресные светодиодные ленты и приспосабливать их под свои нужды. Каждый человек вынужден тратить своё время на поиск украшений, тратить время на их конфигурирование. Это приводит к большим тратам, а так же возникает проблема с изменением анимаций уже после покупки.

Для задачи управления световыми эффектами реализовано не так много приложений для мобильных устройств, что благоприятно сказывается на конкурентоспособности приложения. В основном люди используют обычные гирлянды без возможности управления, либо управление представлено простейшим устройством: кнопка переключения заранее установленных анимаций.

В ООО \enquote{Сампад}, разработка в сфере Интернета вещей является важной статьей дохода. С каждым годом количество проектов и сотрудников, занятых на них, увеличивается в разы. Приложение для управления адресными светодиодными лентами является важной вехой в развитии компании, так как успешное завершение данного проекта способствует ускоренному появлению заказов в данной сфере. Также данный проект важен для компании с точки зрения развития областей разработки, а именно организация отдела по разработке прошивок для плат.

Целью настоящего дипломного проекта является совершенствование процесса взаимодействия пользователей с адресными светодиодными лентами посредством разработки программного средства. Для достижения поставленной цели необходимо решить следующие задачи:
\begin{itemize}
\item изучить особенности организации интернета вещей;
\item проанализировать деятельность ООО «Сампад» в области разработки программного обеспечения для сферы Интернета вещей;
\item разработать программную поддержку для взаимодействия с адресной светодиодной лентой;
\item подготовить технико-экономическое обоснование.
\end{itemize}