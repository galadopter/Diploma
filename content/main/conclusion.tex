\sectioncentered*{Заключение}
\addcontentsline{toc}{section}{Заключение}
%\setcounter{page}{74}
Предметной областью данного дипломного проекта является Интернет вещей и взаимодействие мобильных приложений с устройствами в нем. Во время создания данного дипломного проекта была изучена предметная область, рассмотрены уже существующие системы в Интернете вещей. Была разработана система калибровки адресных светодиодных лент для корректного отображения анимаций. Был проведен анализ деятельности предприятия ООО \enquote{Сампад}, а также рынка Internet of Things. Был проведен анализ процесса управления адресной светодиодной лентой. Было предложено программное средство, которое позволяет не только управлять и создавать эффекты на адресных светодиодных лентах, но и обмениваться ими с другими пользователями. Были сделаны различные модели системы: модели представления и информационные (включая логический и физический уровни). Было сделано технико-экономическое обоснование, рассчитаны затраты на разработку мобильного приложения и рассчитан экономический эффект.

В первой главе была изучена предметная область дипломного проекта. Было рассмотрено понятие Интернета вещей и его связь с мобильными приложениями. Был рассмотрен рынок IoT устройств в Беларуси и индустрия электронного светового оборудования. Также было представлено описание алгоритма распознавания расположения адресных светодиодных лент в пространстве.

Во второй главе был проведен аналих деятельности предприятия ООО \enquote{Сампад} и его участие в создании проектов в сфере IoT. Был освещен прогноз и сделана характеристика рынка IoT в 2019 году. Также было сделано описание бизнес-процесса приобретения и управления адресной светодиодной лентой.

В третьей главе были подробно описаны все аспекты разработки программного обеспечение, а именно мобильного приложения для управления адресными светодиодными лентами. Были разработаны информационная модель и модели представления системы. Был обоснован выбор языка программирования, а также некоторых архитектурных аспектов. Также были описаны некоторые алгоритмы системы.

В четвертой главе было проведено экономическое обоснование проекта. Была сделана характеристика программного продукта. Были рассчитаны смены затраты на разработку мобильного приложения и отпускной цены. Был проведен расчет экономического эффекта от реализации программного продукта. Также был проведен расчет показателей эффективности использования программного продукта.

Следующая основная цель -- внедрение и популяризация программного средства среди пользователей. Параллельно с этим будет производиться дальнейшая его разработка. Будет внедрена поддержка адресных светодиодных лент с различным количеством лампочек (150 и 350). Будет добавлена поддержка различных типов адресных лент (окружности, стены, кубы). Будет развиваться алгоритм калибровки. Кроме того, будут добавляться новые анимации.
