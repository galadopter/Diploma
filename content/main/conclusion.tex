\sectioncentered*{Заключение}
\addcontentsline{toc}{section}{Заключение}
%\setcounter{page}{74}
Предметной областью данного курсового проекта является управление световыми эффектами на гирляндах. Был проведен поиск существующих программные средств этого рода, по его результатам был сделан вывод о несуществовании полных аналогов. Было предложено программное средство, которое позволяет не только управлять и создавать эффекты на гирляндах, но и обмениваться ими с другими пользователями. Так же была разработана система калибровки гирлянд для корректного отображения анимаций. 

На основании проведенного анализа предметной области были выдвинуты требования к программному средству. В качестве технологий разработки были выбраны наиболее современные существующие на данный момент средства, широко применяемые в индустрии. Спроектированное программное средство было успешно протестировано на соответствие спецификации функциональных требований. Уже исходя только из анализа предметной области можно было сделать вывод о целесообразности проектирования и разработки программной системы.

Разработано программное средство, целевой платформой которого является мобильное-приложение и которое поддерживает следующие функции:
\begin{itemize}
	\item отображение анимаций;
	\item создание и управление эффектами;
	\item калибровка гирлянды;
	\item обмен созданными анимациями;
	\item редактирование анимаций;
	\item создание текстовых анимаций.
\end{itemize}

Следующая основная цель -- внедрение и популяризация программного средства среди пользователей. Параллельно с этим будет производиться дальнейшая его разработка. Будет внедрена поддержка гирлянд с различным количеством лампочек (150 и 350). Будет развиваться алгоритм калибровки. Кроме того, будут добавляться новые анимации.
